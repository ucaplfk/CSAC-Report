\section{\label{sec:level1}Introduction}
Highly accurate and stable timing is vital for everyday applications such as navigation, communication and power distribution $^[$\citep{Knappe2007MEMSClocks}$^,$\citep{Mileti2017IntroductionClocks}$^]$. Since the redefinition of the SI second in 1967 using the ground sate caesium (Cs) atomic transition as the frequency reference, a variety of Cs beam atomic clocks were used as the time and frequency standard. However, as atomic trapping and laser cooling techniques became available atomic clock accuracy continued to improve rapidly. In 1998 laser-cooled fountain atomic clocks became the primary frequency standard with the current best accuracy of $\approx10^{-16}$   $^[$\citep{LombardiM.A.andHeavnerT.PandJefferts2007NISTSI}$^,$\citep{Guena2012ProgressLNE-SYRTE}$^]$. The primary standard has small applied adjustments to allow coordinated universal time (UTC) and international timekeeping $^[$\citep{Lombardi2002NISTServices}$^]$. Recently optical lattice and single ion optical clocks exceed fountain clocks performance, with accuracy of $\approx10^{-17}$. Optical clocks are being considered for the SI second redefinition, measurement of the earth's gravitational potential and the possibility of replacing the atomic clocks used in current global navigational satellite systems (GNSSs) as the efforts are made to make these clocks more transportable $^[$\citep{Gill2011WhenSecond}$^,$\citep{Margolis2014TimekeepersFuture}$^,$\citep{Margolis2010OpticalClocks}$^,$\citep{Godun2017TransportableTimes}$^]$.

%Despite the success of continued efforts to improve the performance of atomic clocks it comes at the penalty of size and power consumption. 

Primary standard and optical clocks far exceed the required stability and accuracy for military and commercial applications but they are not portable. Current commercially available chip-scale atomic clocks (CSACs) provide replacement of temperature compensated crystal oscillators (TCXO) for military applications and GNSS receivers with a 88 $\%$ short-term stability improvement $^[$\citep{Fernandez2017CSACPerformance.}$^]$. The availability of the commercial device is the outcome of the CSAC program launched by the Defense Advanced Research Projects Agency (DARPA) in 2001. This program focused on providing lightweight battery-operated atomic clocks for secure communication provided by time-ordered encryption key synchronisation and P(Y)-Code signal acquisition in the presence of local jamming of targeted GNSS receivers $^[$\citep{Lutwak2002TheInterrogation}$^,$\citep{Fruehauf2001FastClock.}$^]$. The aim of the program was to produce an atomic clock with a volume of 1 cm$^3$, stability of 10$^{-11}$ at one hour of integration time and power consumption below 30 mW $^[$\citep{LutwakR.DengJ.RileyW.andVargheseM.2004ThePackage}$^]$. DARPA has since launched the atomic clock with enhanced stability (ACES) program which aims to produce CSAC which meet the previous program requirements clocks but with an extended time stability $^[$\citep{DARPA2016Broad}$^]$. Potential commercial applications of CSACs include underwater sensing, financial trading time-keeping, enhanced GNSS receivers and improved synchronisation for digital communication and power distribution $^[$\citep{Lutwak2011PrinciplesClocks}$^,$\citep{KnappeChip-ScaleClock}$^]$. 



   


 





 













