\section{\label{sec:level1}Outlook}
The ACES program aims to improve the limiting factors of the CSACs operation. These limitations for SA.45s include: the requirement for lengthy calibration each time the device is switched on due to the poor retraceability and inability to predict or correct long-term frequency drifting. In addition, the program intends to investigate pioneering methods of interrogation and atomic confinement $^[$\citep{DARPA2016Broad}$^]$. OEwaves has received funding from the ACES to continue exploring approaches of using a  whispering gallery mode (WGM) micro-resonator to produce a miniature $Rb$ atomic clock $^[$\citep{Maleki2010All-opticalClock}$^,$\citep{Maleki2011All-OpticalClock}$^]$.

The majority of SA.45s CSAC were unable to satisfy the requirements for long-term oceanic deployment due to failure or partial loss of the vacuum. In addition, frequency drifting due component aging is an additional problem when precision timing for the duration of typically 2 year deployments is required. Further improvements to the fabrication of the physics package may enable these devices to withstand underwater conditions. Further testing of the underwater aging of CSACs is necessary to gain a better expectation of frequency drifting so that changes could be made to the device components or correction algorithms could be implemented. The results of replacing TCXO in GNSS receivers with CSAC show considerable benefits particularly in limited GNSS signals. This is due to the short-term CSAC stability, $\sigma_{y}(\tau) < 2\times 10^{-10}/\sqrt{\tau}$ and long-term steering of CSAC using precision timing of GNSS satellite atomic clocks.
